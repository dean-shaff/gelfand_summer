\documentclass[12pt]{article}
\usepackage{times} %typeface
\usepackage{breakcites} %for citations
\usepackage{mathtools} %to write equations
\usepackage[margin=1in]{geometry} %make the margins a little wider
\usepackage{natbib} %this makes it so it writes 'et al'
\usepackage{setspace}
\usepackage{alltt}
\usepackage{upquote}

\setlength\parindent{24pt}
% \usepackage[document]{ragged2e}
\doublespacing

\begin{document}
\bibliographystyle{apj}
% \title{Practice to Theory}
% \author{Dean Shaff}\\
% NYUAD\\
% \date{\today}
% \maketitle
\begin{flushleft}
Experimental astronomy is about fitting models to data. As such, the professor with whom I'm working this summer created a model that generates spectral, electron count, and dynamical data about pulsar wind nebulae from birth up to a specified age. Pulsar wind nebulae (PWNe) are produced by spinning, highly magnetized neutron stars that are often left after stars explode in supernovae. PWNe possess two interesting characteristics: They are the primary sources of high energy EM \cite{mackey2013} radiation in the galaxy, and they produce (or rather, we observe on Earth) roughly equal amounts of antimatter (positron) and matter (electron) emissions. The nature of the data the model produces depends on 18 different parameters, twelve of which are independent. The goal of my work was to determine whether the model could adequately predict the characteristics of the g21.5-0.9 pulsar wind nebula. My professor has already used a modified version of his model to correctly predict observed characteristics of other PWNe \cite{gelfand2011}. My work seeks to confirm the legitimacy of his model (and the theory that backs it) as well as speculate about g21.5-0.9's physical characteristics.\par

\end{flushleft}
\bibliography{citations}

\end{document}